\documentclass[a4paper]{article}
    \linespread{2}
    \def\contra{\tikz[baseline, x=0.22em, y=0.22em, line width=0.032em]\draw (0,2.83)--(2.83,0) (0.71,3.54)--(3.54,0.71) (0,0.71)--(2.83,3.54) (0.71,0)--(3.54,2.83);} 
    \usepackage[english]{babel}
    \usepackage[utf8x]{inputenc}
    
    \usepackage{graphicx}
    \usepackage[colorinlistoftodos]{todonotes}
    \usepackage{amsmath,amsthm, amsfonts,amssymb}
    \title{Homework 0
    PY525}
    
    \author{Edward Chan}
    
    \begin{document}
    \maketitle
    All code are presented in https://github.com/schan3Ed/py525
    \section{Problem 1}
    In problem 1, I created a function name multiple work as a following
    \\ for given $i, j, k$ and $multiply(A, B)$ which i traverse A by rows, j traverse B by columns. They share variable k within the row/column. 
    \\ Each output element in the output matrix calculated by the follow equation
    \\ $\sum_{k}(-1)^{i + j + k} ma[i][k] mb[k][j] $
    \\ The input of this program must be a $n$ by $n$ numpy matrix. The output is also an $n$ by $n$ numpy matrix
    \\ Example output:
    \begin{verbatim}
    [[1 1]
     [1 1]]
    [[1 1]
     [1 1]]
    [[ 0.  0.]
     [ 0.  0.]]
    \end{verbatim}
    \section{Problem 2}
    First part of the problem asked to generate a set of combination with element $1, -1$. Using recursive method, the elements generated are
    \begin{verbatim}
    [[1, -1, 1, -1], [1, -1, 1, 1], [1, -1, -1, 1], 
    [1, -1, -1, -1], [1, 1, -1, 1], [1, 1, -1, -1], 
    [1, 1, 1, -1], [1, 1, 1, 1], [-1, 1, -1, 1], 
    [-1, 1, -1, -1], [-1, 1, 1, -1], [-1, 1, 1, 1], 
    [-1, -1, 1, -1], [-1, -1, 1, 1], [-1, -1, -1, 1], [-1, -1, -1, -1]]
    \end{verbatim}
    The recursive method builds 2 binary trees with height of 4. An output list append the last child of the tree
    \\Second part of the problem uses the combination to calculate the expectation value of $E$ with the equation $<E> = \sum e^{E(p)}E(p)$ with $E(p) = p_1p_2 + p_2p_3 + p_3p_4 + p4_p1$
    \\ The result is: $-436.638675154$
    
    \section{Problem 3}
    This is a shortest path problem with 7 random vertices. Given the small amount of permutations there are. I first generated all possible permutation of all the points recursively. Then I calculated all the distance in each of the set. At last, using python's library to sort the list
    \\Example output: \\
    \includegraphics[width=\linewidth]{src/figure_1.png}
    \includegraphics[width=\linewidth]{src/figure_1-2.png}
    \includegraphics[width=\linewidth]{src/figure_1-3.png}
    \includegraphics[width=\linewidth]{src/figure_1-4.png}
    \\Trivally, if a path from ABCD....Z is the shortest, then ZYX...A will also be the shortest path. However, the output already omitted those repeated path output
    \\And for the fun, here is a comparison of a long path vs a short path\\
    \includegraphics[width=\linewidth]{src/figure_1-5.png}
    \includegraphics[width=\linewidth]{src/figure_1-6.png}
    
    \section{Problem 4}
    I generated a function to calculate the the integral with the trapezoid rule. I then proceed calculating integral with different interval. Due to the precision of trapezoid rule, I can only generate the function at a lower bound. The program will also print out the error value to the user 
    \\Example output:
    \\This is the function with the interval [1, 50] with a trapezoid rule evaluating [0, 0.1] using a 100 steps
    \begin{verbatim}
    Error for this iteration:   0.118370771
    Error for this iteration:   0.119095488
    Error for this iteration:   0.119823157
    Error for this iteration:   0.120553784
    Error for this iteration:   0.121287375
    Error for this iteration:   0.122023936
    Error for this iteration:   0.122763473
    Error for this iteration:   0.123505992
    Error for this iteration:   0.124251499 etc.
    \end{verbatim}
    \includegraphics[width=\linewidth]{src/figure_1-7.png}
    \\Since the value is around 1.5 and the error is around 0.1, its about a 1\% error. So I proceed to generate a more accurate plot given a smaller interval 
    \\My input are [1, 20] with a 200 steps trapezoid rule\\
    \begin{verbatim}
    Error for this iteration:   0.000912673
    Error for this iteration:   0.000926859375
    Error for this iteration:   0.000941192
    Error for this iteration:   0.000955671625
    Error for this iteration:   0.000970299
    Error for this iteration:   0.000985074875
    \end{verbatim}
    \includegraphics[width=\linewidth]{src/figure_1-8.png}
    \\And as you can see, error is at .001\% now which is a more acceptable
    
    \section{Problem 5}
    I created 3 functions to tackle this problem. First functions creates a lattice that filled with zeroes. The Second function convert some to 1 base on given probability. The third function plot the lattice base on the value
    \\Example output:
    \\This is a 20 by 20 lattice with a probability of 30\%\\
    \includegraphics[width=\linewidth]{src/figure_1-9.png}
    \section{Appendix}
    \begin{verbatim}
    
    Problem 1
    import scipy as scipy
    import numpy as numpy
    
    #ma, mb must be a n x n numpy matrix
    def multiply(ma, mb):
        n = ma.ndim
        mc = numpy.zeros((n, n))
        for i in range(0, n):
            for j in range(0, n):
                sum = 0
                for k in range(0, n):
                    mul = 1
                    if (i + j + k) % 2 == 1:
                        mul = -1
                    sum += ma[i, k] * mb[k, j] * mul
                mc[i, j] = sum
        return mc
    
    #example
    ma = numpy.matrix('1 1; 1 1')
    mb = numpy.matrix('1 1; 1 1')
    print ma
    print mb
    print(multiply(ma, mb))
    
    #solved
    
    Problem 2
    
    from collections import deque
    import copy
    import numpy
    
    def rec(head, tail, round_count, all_list, var):
        if(round_count < 1):
            all_list.append(head)
            return
        for i in range(var):
            x = tail.popleft()
            p = list(head)
            p.append(x)
            tail.append(x)
    #       print(p, "    ", round_count, "   ", tail, "   ", i)
            rec(copy.copy(p), copy.copy(tail), round_count - 1, all_list, var)
    
    
    
    tail = deque([1, -1])
    head = deque([])
    round_count = 4
    var = 2
    all_list = []
    rec(head, tail, round_count, all_list, var)
    print("\nAll possible permutation   ", all_list, "\n")
    
    #Ending the permutation code ehre
    sum = 0
    for n in all_list:
        ep = n[0] * n[1] + n[1] * n[2] + n[2] * n[3] + n[3] * n[0]
        sum += numpy.exp(ep * -1) * ep
    
    print(sum)
    
    
    
    Problem 3
    
    import copy
    import random
    from collections import deque
    import math
    import matplotlib.pyplot as plt
    
    def rec(head, tail, all_list):
        if(len(tail) < 1):
            all_list.append(head)
            return
        for i in range(len(tail)):
            x = tail[i]
            po = copy.copy(tail)
            po.pop(i)
            p = list(head)
            p.append(x)
          #  tail.append(x)
    #       print(p, "    ", round_count, "   ", tail, "   ", i)
            rec(copy.copy(p), po, all_list)
    
    x = random.random()
    pairs = []
    #print(x)
    
    for i in range(7):
        x = random.random()
        y = random.random()
        x = round(x, 2)
        y = round(y, 2)
        pair = []
        pair.append(x)
        pair.append(y)
        pairs.append(pair)
    
    perm_pairs = []
    
    
    #print("This is a set of paris    \n", pairs)
    
    rec([], pairs, perm_pairs)
    
    
    
    
    for pairs in perm_pairs:
        sum = 0
        for i in range(0, len(pairs) - 1):
         #   print(pairs[i])
            x1 = pairs[i][0]
            y1 = pairs[i][1]
            x2 = pairs[i + 1][0]
            y2 = pairs[i + 1][1]
      #      print(pairs[i])
            sum += math.sqrt(math.pow((x1 - x2), 2) + math.pow((y1 - y2), 2))
     #       print(sum)
        pairs.insert(0, sum)    
    perm_pairs.sort()
    pairs = perm_pairs[0][1:]
    
    for i in range(0, 10, 2):
        pairs = perm_pairs[i][1:]
        x = []
        y = []
        for pair in pairs:
            x.append(pair[0])
            y.append(pair[1])
            plt.plot(x, y, marker="o", markerfacecolor="r",)
        plt.title("The distance of this route: " + str(perm_pairs[i][0]))
        plt.show()   
        
    
    #print(perm_pairs[0:10])
    
    Problem 4
    
    import numpy
    import math
    import matplotlib.pyplot as plt
    
    
    def fx(p):
        return math.sin(p) / math.sqrt(p)
    
    
    def trap(N, b, a):
        h = float(b - a) / N
        sum = 0
        for k in range(1, N - 1):
            xk0 = a + k * h
            xk1 = a + (k + 1) * h
            sum += fx(xk0) + fx(xk1)
        print "Error for this iteration:  ", math.pow(h, 3)
       # print sum
        return float(h) / 2 * sum
    
    for i in range(1, 200):
        plt.scatter(i * 0.1, trap(200, i * 0.1, 0 ))
      #  print trap(10, i, 0 )
    plt.show()
    
    problem 5
    
    import matplotlib.pyplot as plt
    import random
    
    
    def prob(n, l):
        for i in range(n):
            for j in range(n):
                x = random.random()
                if(x < 0.3333333):
                    l[i][j] = 1
    
    def lattice(n, l):
        for i in range(n):
            l.append([])
            for j in range(n):
                l[i].append(0)
    
    def plot(n, l):
        for i in range(n):
            for j in range(n):
                if (l[i][j] == 1) :
                    str = "red"
                else:
                    str = "none"
                plt.scatter(i, j, marker="o", facecolors=str)
        plt.show()
    
    l = []
    lattice(20, l)
    prob(20, l)
    plot(20, l)
    
    
    
    
    \end{verbatim}
    \end{document}